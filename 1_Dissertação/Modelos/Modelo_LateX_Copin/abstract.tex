The search for new music, e.g. songs tracks, albums or artists, is a central aspect in the habit of people when it comes to music. And this pursuit increased mainly because of the large amount of  available music and with easy access provided by the advance of technologies like Last.FM, Sportify, Youtube, Itunes. However, due to this high music availability is not always easy to discover relevant novelties. This study attempts to expand the studies about music novelties by investigating how the music preferences of listeners are affected by two different aspects of novel artists: familiarity (how much the listener knows other artists similar to novelty) and popularity (how this artist is known by listeners in general ). The study supports this multidimensional view of novelty, which is a richer view and it enables the improvement of tools that support the discovery of music novelties for listeners, as recommender systems,  websites, forums, etc.. 
We collected and analyzed historical data from Last.fm users, a popular online music discovery service. The results suggest that there is not a general preference for some aspect of novelty. Listeners tend to form groups based on the preferences for the novelty aspects. These results suggest a specific treatment for these groups of listeners, e.g. a recommendation system considering these preferences. Another study performed  compares the listeners preferences by aspects of both novelty artists and artists already known. This study showed that the listeners preferences for these two spheres are different, where listeners tend to form groups based on these different preferences. This result implies that the scope of novelty and the scope of what is already known should not be treated the same way.