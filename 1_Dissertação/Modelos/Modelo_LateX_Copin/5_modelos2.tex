	\chapter{Modelos} \label{cap:modelos}
	
	Após a coleta e filtragem dos dados, utilizamos 3 conceitos para representar as entidades envolvidas no estudo: modelo de perfil do ouvinte, características das novidades e métricas de relevância. Primeiro, foi construindo o modelo do perfil do ouvinte, com o intuito de: gerar uma representação visual do que foi escutado pelo ouvinte; viabilizar o cálculo da familiaridade de um artista para um ouvinte; e gerar uma métrica de ecleticidade. Segundo, foram modeladas as características da novidade a serem utilizadas nos experimentos - familiaridade e popularidade. Por fim, foram modeladas duas métricas que refletem a preferência do ouvinte para um artista, ou a relevância deste artista para o ouvinte, durante um período de tempo - a atenção total e o período de atenção.
 
	\section{Perfil musical do ouvinte}\label{sec:perfil}
 

	\section{Características das novidades}  
	
