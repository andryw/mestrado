A busca por novidades musicais, sejam elas m�sicas, �lbuns ou artistas, � um aspecto central no h�bito das pessoas quando se trata de m�sica. E esta procura aumentou principalmente por causa da grande quantidade de m�sica dispon�vel e com f�cil acesso proporcionado pelo avan�o de tecnologias como Last.FM, Sportify, Youtube, Itunes, entre outros. Por�m, devido a esta grande disponibilidade, nem sempre � f�cil a descoberta de novidades que sejam relevantes. Para resolver este problema, muitos esfor�os foram elaborados. O presente trabalho tenta expandir estes esfor�os tratando a novidade de maneira multidimensional, de acordo com dois aspectos: familiaridade (o quanto o ouvinte conhece outras m�sicas/ artistas similares � novidade) e popularidade (o qu�o essa m�sica / artista � conhecida pelos ouvintes em geral). O presente trabalho corrobora esta vis�o multidimensional da novidade, que � uma vis�o mais rica e que pode aperfei�oar ferramentas que d�o suporte a descoberta de novidades para ouvintes, como sistemas de recomenda��o, sites, f�runs, etc.
Desta maneira analisamos as prefer�ncias dos ouvintes por artistas com novidade (artistas que nunca foram escutados anteriormente pelo ouvinte) baseadas nestes dois aspectos. Para isso foi estudado
os h�bitos de escuta dos usu�rio do Last.FM, rede social musical que registra o que os usu�rios escutam. Os resultados
sugerem que n�o existe uma prefer�ncia geral dos ouvintes por algum aspecto das novidade. Os ouvintes tendem a formar
grupos baseados nas prefer�ncias pelos aspectos das novidades. Estes resultados sugerem um tratamento espec�fico
para estes grupos de ouvintes, como um sistema de recomenda��o que leve em conta estas prefer�ncias. Outro estudo
realizado neste trabalho compara as prefer�ncias dos ouvintes pelos aspectos tanto dos artistas com novidade quanto dos artistas
j� conhecidos. Este estudo apontou que as prefer�ncias dos ouvintes para estes dois �mbitos s�o diferentes, onde os
ouvintes tendem a formar grupos baseados nestas diferen�as de prefer�ncias. Este resultado implica que o �mbito
das novidade e o �mbito do que j� se conhece n�o deve ser tratado da mesma maneira.
