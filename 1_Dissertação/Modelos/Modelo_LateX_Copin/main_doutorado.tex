%%%%%%%%%%%%%%%%%%%%%%%%%%%%%%%%%%%%%%%%%%%%%%%%%%%%%%%%%%%%%%%%%%%%%%%%%%%%%%%%
%%
%% Para utilizar ese modelo sao necessarios os seguintes arquivos:
%%
%% copin.cls
%% copin.sty
%% doutor.sty
%%
%%%%%%%%%%%%%%%%%%%%%%%%%%%%%%%%%%%%%%%%%%%%%%%%%%%%%%%%%%%%%%%%%%%%%%%%%%%%%%%%

\documentclass[a4paper,titlepage]{copin}
\usepackage[portuges,english]{babel}
\usepackage{copin,doutor,epsfig}
\usepackage{times}

%-------------------------- Para usar acentuacaoo em sistemas ISO8859-1 ------------------------------------
% Se estiver usando o Microsoft Windows ou linux com essa codificacao, descomente essa linhas abaixo
% e comente as linhas referentes ao UTF8
\usepackage[latin1]{inputenc} % Usar acentuacao em sistemas ISO8859-1, comentar a linha com  \usepackage[utf8x]{inputenc}
%-----------------------------------------------------------------------------------------------------

%-------------------------- Para usar acentuacao em sistemas UTF8 ------------------------------------
% Para a maior parte das distribuicoes linux, usar a opcao utf8x (lembrar de comentar as linha referente a ISO8859-1 acima)
\usepackage{ucs}
%\usepackage[utf8x]{inputenc}
%\usepackage[utf8]{inputenc}
\usepackage[T1]{fontenc}
%-----------------------------------------------------------------------------------------------------

\usepackage{fancyheadings}
\usepackage{graphicx}
\usepackage{longtable} %tabelas longas, para tabelas que ultrapassam uma pagina
%\input{psfig.sty}


% ----------------- Para inserir codigo fonte de linguagens de programacao no documento -------------
\usepackage{listings}
\lstset{numbers=left,
stepnumber=1,
firstnumber=1,
%numberstyle=\tiny,
extendedchars=true,
breaklines=true,
frame=tb,
basicstyle=\footnotesize,
stringstyle=\ttfamily,
showstringspaces=false
}
\renewcommand{\lstlistingname}{C\'odigo Fonte}
\renewcommand{\lstlistlistingname}{Lista de C\'odigos Fonte}
% ---------------------------------------------------------------------------------------------------

\selectlanguage{portuges}
\sloppy



\begin{document}



%%%%%%%%%%%%%%%%%%%%%%%%%%%%%%%%%%%%%%%%%%%%%%%%%%%%%%%%%%%%%%%%%%%%%%%%%%%%%%%%
\Titulo{T�tulo de sua Tese}
\Autor{Nome do Aluno}
\Data{01/06/2007}
\Area{Ci�ncia da Computa��o}
\Pesquisa{Linha de Pesquisa}
\Orientadores{Nome Do Orientador  \\
	 (Orientador)}

\newpage
\cleardoublepage

\PaginadeRosto

\newpage
\cleardoublepage

%%%%%%%%%%%%%%%%%%%%%%%%%%%%%%%%%%%%%%%%%%%%%%%%%%%%%%%%%%%%%%%%%%%%%%%%%%%%%%%%
\begin{resumo} 
A busca por novidades musicais, sejam elas m�sicas, �lbuns ou artistas, � um aspecto central no h�bito das pessoas quando se trata de m�sica. E esta procura aumentou principalmente por causa da grande quantidade de m�sica dispon�vel e com f�cil acesso proporcionado pelo avan�o de tecnologias como Last.FM, Sportify, Youtube, Itunes, entre outros. Por�m, devido a esta grande disponibilidade, nem sempre � f�cil a descoberta de novidades que sejam relevantes. Para resolver este problema, muitos esfor�os foram elaborados. O presente trabalho tenta expandir estes esfor�os tratando a novidade de maneira multidimensional, de acordo com dois aspectos: familiaridade (o quanto o ouvinte conhece outras m�sicas/ artistas similares � novidade) e popularidade (o qu�o essa m�sica / artista � conhecida pelos ouvintes em geral). O presente trabalho corrobora esta vis�o multidimensional da novidade, que � uma vis�o mais rica e que pode aperfei�oar ferramentas que d�o suporte a descoberta de novidades para ouvintes, como sistemas de recomenda��o, sites, f�runs, etc.
Desta maneira analisamos as prefer�ncias dos ouvintes por artistas com novidade (artistas que nunca foram escutados anteriormente pelo ouvinte) baseadas nestes dois aspectos. Para isso foi estudado
os h�bitos de escuta dos usu�rio do Last.FM, rede social musical que registra o que os usu�rios escutam. Os resultados
sugerem que n�o existe uma prefer�ncia geral dos ouvintes por algum aspecto das novidade. Os ouvintes tendem a formar
grupos baseados nas prefer�ncias pelos aspectos das novidades. Estes resultados sugerem um tratamento espec�fico
para estes grupos de ouvintes, como um sistema de recomenda��o que leve em conta estas prefer�ncias. Outro estudo
realizado neste trabalho compara as prefer�ncias dos ouvintes pelos aspectos tanto dos artistas com novidade quanto dos artistas
j� conhecidos. Este estudo apontou que as prefer�ncias dos ouvintes para estes dois �mbitos s�o diferentes, onde os
ouvintes tendem a formar grupos baseados nestas diferen�as de prefer�ncias. Este resultado implica que o �mbito
das novidade e o �mbito do que j� se conhece n�o deve ser tratado da mesma maneira.

\end{resumo}

\newpage
\cleardoublepage

%%%%%%%%%%%%%%%%%%%%%%%%%%%%%%%%%%%%%%%%%%%%%%%%%%%%%%%%%%%%%%%%%%%%%%%%%%%%%%%%
\begin{summary}
The search for new music, e.g. songs tracks, albums or artists, is a central aspect in the habit of people when it comes to music. And this pursuit increased mainly because of the large amount of  available music and with easy access provided by the advance of technologies like Last.FM, Sportify, Youtube, Itunes. However, due to this high music availability is not always easy to discover relevant novelties. This study attempts to expand the studies about music novelties by investigating how the music preferences of listeners are affected by two different aspects of novel artists: familiarity (how much the listener knows other artists similar to novelty) and popularity (how this artist is known by listeners in general ). The study supports this multidimensional view of novelty, which is a richer view and it enables the improvement of tools that support the discovery of music novelties for listeners, as recommender systems,  websites, forums, etc.. 
We collected and analyzed historical data from Last.fm users, a popular online music discovery service. The results suggest that there is not a general preference for some aspect of novelty. Listeners tend to form groups based on the preferences for the novelty aspects. These results suggest a specific treatment for these groups of listeners, e.g. a recommendation system considering these preferences. Another study performed  compares the listeners preferences by aspects of both novelty artists and artists already known. This study showed that the listeners preferences for these two spheres are different, where listeners tend to form groups based on these different preferences. This result implies that the scope of novelty and the scope of what is already known should not be treated the same way.
\end{summary}

\newpage
\cleardoublepage

%%%%%%%%%%%%%%%%%%%%%%%%%%%%%%%%%%%%%%%%%%%%%%%%%%%%%%%%%%%%%%%%%%%%%%%%%%%%%%%%
\begin{agradecimentos}
Agrade�o a todos que me ajudaram at� aqui.
\end{agradecimentos}

\clearpage

%%%%%%%%%%%%%%%%%%%%%%%%%%%%%%%%%%%%%%%%%%%%%%%%%%%%%%%%%%%%%%%%%%%%%%%%%%%%%%%%
%% Definicao do cabecalho: secao do lado esquerdo e numero da pagina do lado direito
\pagestyle{fancy}
\addtolength{\headwidth}{\marginparsep}\addtolength{\headwidth}{\marginparwidth}\headwidth = \textwidth
\renewcommand{\chaptermark}[1]{\markboth{#1}{}}
\renewcommand{\sectionmark}[1]{\markright{\thesection\ #1}}\lhead[\fancyplain{}{\bfseries\thepage}]%
	     {\fancyplain{}{\emph{\rightmark}}}\rhead[\fancyplain{}{\bfseries\leftmark}]%
             {\fancyplain{}{\bfseries\thepage}}\cfoot{}

%%%%%%%%%%%%%%%%%%%%%%%%%%%%%%%%%%%%%%%%%%%%%%%%%%%%%%%%%%%%%%%%%%%%%%%%%%%%%%%%
\selectlanguage{portuges}

\Sumario
\ListadeSimbolos
\listoffigures
\listoftables
\lstlistoflistings %lista de codigos fonte - Para inserir a listagem de codigos fonte
\newpage
\cleardoublepage

\Introducao


%%%%%%%%%%%%%%%%%%%%%%%%%%%%%%%%%%%%%%%%%%%%%%%%%%%%%%%%%%%%%%%%%%%%%%%%%%%%%%%%
%
% Hifenizacao - Colocar lista de palavras que nao devem ser separadas e que 
% nao estao no dicionario portugues.
% As palavras do dicionario portugues ja sao separadas corretamente pelo lateX
%
\hyphenation{ Hardware Software etc  }


%%%%%%%%%%%%%%%%%%%%%%%%%%%%%%%%%%%%%%%%%%%%%%%%%%%%%%%%%%%%%%%%%%%%%%%%%%%%%%%%
%% A partir daqui coloque seus capitulos. Sugere-se que eles sejam inseridos com o comando \input
%% Da seguinte maneira:
%%
%% \input{cap1} 
%% \input{cap2}
\input{cap1}


%%%%%%%%%%%%%%%%%%%%%%%%%%%%%%%%%%%%%%%%%%%%%%%%%%%%%%%%%%%%%%%%%%%%%%%%%%%%%%%%
%% BIbliografia
%% Coloque suas referencias no arquivo ref.bib e descomente as proximas duas linhas

\bibliographystyle{plain} % estilo de bibliografia   plain,unsrt,alpha,abbrv.
\bibliography{ref} % arquivos com as entradas bib.

%%%%%%%%%%%%%%%%%%%%%%%%%%%%%%%%%%%%%%%%%%%%%%%%%%%%%%%%%%%%%%%%%%%%%%%%%%%%%%%%
%% Apendice
% Caso seja necessario algum apendice, descomente a proxima linha.

\appendix
\chapter{Ap�ndices}
%%%%%%%%%%%%%%%%%%%%%%%%%%%%%%%%%%%%%%%%%%%%%%%%%%%%%%%%%%%%%%%%%%%%%%%%%%%%%%%%

\end{document}